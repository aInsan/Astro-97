\documentclass{article}
\usepackage{graphicx}
\usepackage{amsmath}
\usepackage{caption}

\title{Simulated Experiment Report}
\author{Insan Adhikari}
\date{}

\begin{document}

\maketitle

\section{Introduction}
In this fictional experiment, we studied the relationship between light intensity and distance from a source in a controlled environment. The experiment was set up using a single light bulb and a sensor that measured light intensity at various distances.

\section{Data and Analysis}
We recorded the intensity of light at different distances from the bulb. The data is shown in Table~\ref{tab:data}.

\begin{table}[h]
\centering
\caption{Light intensity measurements at different distances}
\label{tab:data}
\begin{tabular}{|c|c|}
\hline
Distance (m) & Intensity (lux) \\
\hline
1.0 & 1000 \\
2.0 & 250 \\
3.0 & 111 \\
4.0 & 62.5 \\
\hline
\end{tabular}
\end{table}

Using the inverse square law, we expect the intensity $I$ to follow the equation
\begin{equation}
I_d = \frac{L}{4\pi d^2}
\label{eq:intensity}
\end{equation}
where $I_d$ is the intensity at distance $d$, $L$ is the luminosity of the source, and $\pi$ is the mathematical constant.

\section{Results}
We plotted the measured intensity against distance and compared it to the expected inverse square trend. Figure~\ref{fig:results} shows this comparison.

\begin{figure}[h]
\centering
\includegraphics[width=0.6\textwidth]{result_plot.png}
\caption{Measured light intensity vs. distance, with curve showing inverse square law prediction}
\label{fig:results}
\end{figure}

The data closely follows the theoretical prediction, confirming the inverse square relationship.

\section{Conclusions}
This experiment demonstrated the inverse square law governing light intensity. The measurements and the curve in Figure~\ref{fig:results} agree with the model shown in Equation~\ref{eq:intensity}. This confirms the expected physical behavior of light propagation from a point source.

\end{document}
